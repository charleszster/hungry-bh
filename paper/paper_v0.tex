\documentclass[english, apj]{emulateapj}
\usepackage[T1]{fontenc}
\usepackage[latin9]{inputenc}
\usepackage{array}
\usepackage{rotating}
\usepackage{units}
\usepackage{textcomp}
\usepackage{amsmath}
\usepackage{amsbsy}
\usepackage{amstext}
\usepackage{graphicx}
\usepackage{url}
\usepackage{babel}
\usepackage[backref,breaklinks,colorlinks,citecolor=blue]{hyperref}
\usepackage[all]{hypcap}

\makeatletter

\newcommand{\msun}{\mbox{$\,{\rm M}_\odot$}}

\sloppy

\providecommand{\tabularnewline}{\\}




\begin{document}


\title{Some Title}


\author{Charles Zivancev\altaffilmark{1}, Andreas H.W. K\"upper\altaffilmark{1,2}, Jeremiah P. Ostriker \altaffilmark{1}}
\altaffiltext{1}{Department of Astronomy, Columbia University, 550 West 120th Street, New York, NY 10027, USA}
\altaffiltext{2}{Hubble Fellow}
\email{Correspondence to: akuepper@astro.columbia.edu}




\begin{abstract}
Here Charles can write an abstract.
\end{abstract}


\keywords{dark matter --- Galaxy: halo --- Galaxy: kinematics and dynamics}




\section{Introduction}\label{sec:introduction}
Just some introduction
....



\section{Some name}\label{sec:somename}

%figure example
\begin{figure}
\centering
%\includegraphics[width=0.45\textwidth]{plots/halflight.png}
\caption{...}
\label{halflight}
\end{figure}


%table example
\begin{table*}
 \centering
 \label{tab:profiles}
 \caption{Ellipse, 2D S\'ersic, and 2D Gauss fits to the $N$-body models}
 \begin{tabular}{cc|ccc|ccc|cc}
 Model & $r_{eff,0}$ [pc] & $a$ [pc] &  $a$ [deg] & $\epsilon$ & S\'ersic $n$ & S\'ersic $r_{eff}$ [deg] & S\'ersic $\epsilon$ & Gauss $\sigma_1$ [deg] & Gauss $\epsilon$\\
 \hline
1 & 30 & 29 & 0.012 & 0.05 & 1.1 & 0.013 & 0.09 & 0.010  & 0.05 \\
2 & 37 & 35 & 0.014 & 0.05 & 1.1 & 0.020 & 0.02 & 0.013  & 0.04 \\
3 & 45 & 53 & 0.022 & 0.11 & 1.5 & 0.047 & 0.13 & 0.027  & 0.15 \\
4 & 52 & 128 & 0.052 & 0.40 & 1.2 & 0.093 & 0.15 & 0.054  & 0.26 \\
5 & 60 & 142 & 0.058 & 0.03 & 1.4 & 0.219 & 0.39 & 0.108  & 0.39 \\
6 & 67 & 310 & 0.123 & 0.48 & 1.4 & 0.395 & 0.57 & 0.213  & 0.61 \\
\end{tabular}
\end{table*}





\subsection{...}
We want to cite \citet{Aarseth03}, who is referenced in Sec.~\ref{sec:introduction}. He came up with the equation
\begin{equation}
M(r) = \frac{GM}{r^2}.
\end{equation}


\section{Conclusions}\label{sec:conclusions}

...




\section*{Acknowledgements}

...



\bibliographystyle{apj}
\begin{thebibliography}{}

\bibitem[\protect\citeauthoryear{Aarseth}{2003}]{Aarseth03}
Aarseth, S.~J., 2003, Gravitational N-Body Simulations (Cambridge University Press)


\end{thebibliography}
\end{document}



